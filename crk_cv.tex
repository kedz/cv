%%%%%%%%%%%%%%%%%%%%%%%%%%%%%%%%%%%%%%%%%
% Classicthesis-Styled CV
% LaTeX Template
% Version 1.0 (22/2/13)
%
% This template has been downloaded from:
% http://www.LaTeXTemplates.com
%
% Original author:
% Alessandro Plasmati
%
% License:
% CC BY-NC-SA 3.0 (http://creativecommons.org/licenses/by-nc-sa/3.0/)
%
%%%%%%%%%%%%%%%%%%%%%%%%%%%%%%%%%%%%%%%%%

%----------------------------------------------------------------------------------------
%	PACKAGES AND OTHER DOCUMENT CONFIGURATIONS
%----------------------------------------------------------------------------------------

\documentclass{scrartcl}

\reversemarginpar % Move the margin to the left of the page 

\newcommand{\MarginText}[1]{\marginpar{\raggedleft\itshape\small#1}} % New command defining the margin text style

\usepackage[nochapters]{classicthesis} % Use the classicthesis style for the style of the document
\usepackage[LabelsAligned]{currvita} % Use the currvita style for the layout of the document

\renewcommand{\cvheadingfont}{\LARGE\color{Maroon}} % Font color of your name at the top

\usepackage{hyperref} % Required for adding links	and customizing them
\hypersetup{colorlinks, breaklinks, urlcolor=Maroon, linkcolor=Maroon} % Set link colors

\newlength{\datebox}\settowidth{\datebox}{Spring 2011} % Set the width of the date box in each block

\newcommand{\NewEntry}[3]{\noindent\hangindent=2em\hangafter=0 \parbox{\datebox}{\small \textit{#1}}\hspace{1.5em} #2 #3 % Define a command for each new block - change spacing and font sizes here: #1 is the left margin, #2 is the italic date field and #3 is the position/employer/location field
\vspace{0.5em}} % Add some white space after each new entry

\newcommand{\Description}[1]{\hangindent=2em\hangafter=0\noindent\raggedright\footnotesize{#1}\par\normalsize\vspace{1em}} % Define a command for descriptions of each entry - change spacing and font sizes here

%----------------------------------------------------------------------------------------

\begin{document}

\thispagestyle{empty} % Stop the page count at the bottom of the first page

%----------------------------------------------------------------------------------------
%	NAME AND CONTACT INFORMATION SECTION
%----------------------------------------------------------------------------------------

\begin{cv}{\spacedallcaps{Chris Kedzie}}\vspace{1.5em} % Your name

\noindent\spacedlowsmallcaps{Personal Information}\vspace{0.5em} % Personal information heading

\NewEntry{}{\textit{Made in California,}}{January, 29$^{th}$ 1986} % Birthplace and date

\NewEntry{web}{\href{http://www.cs.columbia.edu/~kedzie}{www.cs.columbia.edu/$\sim$kedzie}} % Email address

\NewEntry{email}{\href{mailto:kedzie@cs.columbia.edu}{kedzie@cs.columbia.edu}} % Email address

\NewEntry{github}{\href{https://github.com/kedz}{https://github.com/kedz}} % Personal website

%\NewEntry{phone}{(H) +1 (000) 111 1111\ \ $\cdotp$\ \ (M) +1 (000) 111 1112} % Phone number(s)
\NewEntry{phone}{+1 (925) 323 1837} % Phone number(s)

\vspace{1em} % Extra white space between the personal information section and goal

\noindent\spacedlowsmallcaps{About}\vspace{1em} % Goal heading, could be used for a quotation or short profile instead

\Description{I am currently a third year Ph.D. student in the Department of 
 Computer Science at Columbia University, working in the field of Natural 
 Language Processing under Prof. \textsc{Kathleen McKeown}. My research has 
 focused on applied machine learning for large scale, streaming news 
 summarization. I am currently working on text generation for multi-document 
 and streaming news summarization. In particular, I am interested in recent 
 advances in deep learning for text generation, and their application to text 
 summarization where there is an opportunity to jointly model the content 
 selection, planning, and realization phases that are typically designed as a 
 pipeline. 
 
 %To that end I am focusing on
 %learning hierarchical semantic representations of text that allow generation
 %to be composed at the phrase level.
 
% Many interesting problems
% stand in the way of this ultimate goal. For instance, annotated data for the 
% multi-document summarization task is usually quite small (few hundreds), 
% requiring some combination of unsupervised learning, transfer learning, and
% regularization. The possible output space is also quite large and most
% generative models of text struggle to produce coherent outputs 
% amount of labeled data is usually much smaller than what is typically
% expected 
% coherence, with neural network sequence models to generate more natural, 
 %informative, and reliable summaries
 }\vspace{2em} % Goal text

%----------------------------------------------------------------------------------------
%	PUBLICATIONS
%----------------------------------------------------------------------------------------

%\spacedlowsmallcaps{Publications}\vspace{1em}

%\NewEntry{}{Chris Kedzie, Kathleen McKeown, and Fernando Diaz. Columbia University at TREC 2014. In
%\textit{Online Proceedings of the Fourteenth Text REtrieval Conference (TREC2014)
%, 2014}}

%\NewEntry{Feb. 2015}{Columbia U. at TREC: Temporal Summarization}

%\Description{\MarginText{Temporal Summarization Track, TREC 2014}Abstract: In

%\newline
%Joint work with \textsc{Fernando Diaz} \& \textsc{Kathleen McKeown}.
%}

%\NewEntry{Feb. 2014}{Aug. 2014}

%\Description{\MarginText{Proceedings of the Bloomberg Workshop on Social Good, KDD, August 2014.}Abstract: We have developed a text summarization system that
%can generate summaries over time from web crawls on disasters. We show that
%our method of identifying exemplar sentences for a summary using affinity
%propagation clustering produces better summaries than clustering based on
%K-medoids as measured using Rouge on a small set of examples. A key component
%of our approach is the prediction of salient information using event related
%features based on location, temporal changes in topic, and two different
%language models.\newline
%Joint work with \textsc{Fernando Diaz} \& \textsc{Kathleen McKeown}.
%}



\spacedlowsmallcaps{Publications}\vspace{1em}



\NewEntry{July 2016}{Chris Kedzie, Fernando Diaz, and Kathleen McKeown. 
    \textit{Real-Time Web Scale Event Summarization Using Sequential Decision 
    Making}}

\Description{\MarginText{Proceedings of the 25th International Joint Conference on Artificial Intelligence} Abstract:
We present a system based on sequential decision making for the online 
 summarization of massive document streams, such as those found on the web.  
 Given an event of interest (e.g. ``boston marathon bombing''), our system is 
 able to filter the stream for relevance and produce a series of short text 
 updates describing the event as it unfolds over time. Unlike previous work, 
 our approach is able to jointly model the relevance, comprehensiveness, 
 novelty, and timeliness required by time-sensitive queries.  We demonstrate 
 a 28.3\% improvement in summary $F_1$ and a 43.8\% improvement time-sensitive 
 $F_1$ metrics. }




\NewEntry{July 2015}{Chris Kedzie, Kathleen McKeown, and Fernando Diaz. \textit{Predicting Salient Updates for Disaster Summarization}}

\Description{\MarginText{Proceedings of the 53nd Annual Meeting of the Association for Computational Linguistics} Abstract:
During crises such as natural disasters or other human tragedies, 
information needs of both civilians and responders often require urgent, 
specialized treatment. Monitoring and summarizing a text stream during such 
an event remains a difficult problem. We present a system for update 
summarization which predicts the salience of sentences with respect to an 
event and then uses these predictions to directly bias a clustering algorithm 
for sentence selection, increasing the quality of the updates. We use novel, 
disaster-specific features for salience prediction, including 
geo-locations and language models representing the language of disaster. 
Our evaluation on a standard set of retrospective events using ROUGE shows 
that salience prediction provides a significant improvement over other 
approaches.}

\pagebreak 

\spacedlowsmallcaps{Talks}\vspace{1em}

\NewEntry{Oct. 2016}{Machine Learning and Friends Lunch: Real-Time Web Scale Event Summarization Using
Sequential Decision Making}

\Description{\MarginText{UMass Amherst}I presented our recent summarization research at the CS department's weekly machine learning talk. 
See associated paper abstract.}




\NewEntry{July. 2016}{Real-Time Web Scale Event Summarization Using
Sequential Decision Making}

\Description{\MarginText{International Joint Conference on Artificial Intelligence}I presented our long-paper submission at the 2016 meeting of the 
International Joint Conference on Artificial Intelligence in New York City. 
See associated paper abstract.}


\NewEntry{Nov. 2015}{Learning 2 Summarize: TREC 2015}

\Description{\MarginText{Temporal Summarization Track, TREC 2015}Abstract: In
this talk, I present an overview of our participation in the temporal 
summarization track at the 2015 Text Retrieval Conference. 
%Using a machine
%learning based approach to the temporal summarization task is difficult 
%because 
Most of the available training data for this task consists of static judgments
on returned updates, making it difficult to make use of sequential predictions
in a learned model. I show how we used learning based search
 (SEARN, Learning2Search, LOLS) to sample 
realistic runs over the training streams and learn from dynamic features
like previous update decisions and rolling stream observations.  Our 
resulting system is able to build an event summary in an online fashion
avoiding latency penalties while still outperforming retrospective 
approaches (e.g. clustering). 
\newline
Joint work with \textsc{Fernando Diaz}.
}




\NewEntry{July 2015}{Predicting Salient Updates for Disaster Summarization}

\Description{\MarginText{Association of Computational Linguistics} I presented our long-paper
submission at the 2015 meeting of the Association for Computational Linguistics
in Beijing, China. See associated paper 
abstract.}





\NewEntry{Nov. 2014}{Columbia U. at TREC: Temporal Summarization}

\Description{\MarginText{Temporal Summarization Track, TREC 2014}Abstract: In
this talk, I present an overview of our participation in the temporal 
summarization track at the 2014 Text Retrieval Conference. Our 
submission was one of the top overall submissions for this track.
Our performance gain came largely from our precision in the 
summary update selection stage; I outline the details of our salience
regression model and affinity propagation clustering architecture, including
their effect on our scores. I also address our current system shortcomings,
especially our inability to explicitly control for redundancy.
\newline
Joint work with \textsc{Fernando Diaz} \& \textsc{Kathleen McKeown}.
}


\NewEntry{Aug. 2014}{Summarizing Disasters Over Time}

\Description{\MarginText{Bloomberg Data Frameworks Track, KDD 2014}Abstract: We have developed a text summarization system that
can generate summaries over time from web crawls on disasters. We show that
our method of identifying exemplar sentences for a summary using affinity
propagation clustering produces better summaries than clustering based on
K-medoids as measured using Rouge on a small set of examples. A key component
of our approach is the prediction of salient information using event related
features based on location, temporal changes in topic, and two different
language models.\newline
Joint work with \textsc{Fernando Diaz} \& \textsc{Kathleen McKeown}.
}

\vspace{1em} % Extra space between major sections


\noindent\spacedlowsmallcaps{Demos}\vspace{1em}

\NewEntry{April 2016}{Monitoring Large Scale Disasters, 
\textsc{Data Science Day}}

\Description{\MarginText{Columbia University's Data Science Institute}
During crises such as natural disasters or other human tragedies, information   needs of both civilians and responders often require urgent, specialized   treatment. Monitoring and summarizing important information during such an   event remains a difficult problem. We present a system for monitoring online   news for such disasters. Given a query: e.g. "Hurricane Sandy," our system   analyzes the web, and produces a sequence of updates, brief textual   descriptions about the current state of the event, as that event unfolds   over time.    We use novel, disaster-specific features for generating updates,    including geo-locations and language models representing the language of    disaster.   Our demo will allow users to see updates generated for   pre-run queries including: Hurricane Sandy, the Boston Marathon bombing,   and 40 other large scale disasters.  

}





\vspace{1em} % Extra space between major sections
%\pagebreak 
\noindent\spacedlowsmallcaps{Doctoral Consortium}\vspace{1em}

\NewEntry{July 2016}{Extractive and Abstractive Event Summarization over Streaming Web Text, \textsc{25th International Joint Conference on Artificial Intelligence}}

\Description{
~~~~}
%------------------------------------------------

\vspace{1em} % Extra space between major sections

\noindent\spacedlowsmallcaps{Summer Schools}\vspace{2em}

\NewEntry{August 2016}{Deep Learning Summer School at the University of Montreal}

\Description{
~~~~}
%------------------------------------------------

\vspace{1em} % Extra space between major sections



\noindent\spacedlowsmallcaps{Departmental Activities}\vspace{1em}

\NewEntry{Sept. 2014 -- Present}{Organizer, Columbia NLP Talks}

\Description{Coordinate and plan internal and visiting speakers to the NLP 
group at Columbia.
}
\NewEntry{June 2016 -- Present}{Organizer, Columbia NLP Reading Group}

\Description{Coordinate and plan weekly reading group on
    current research in 
NLP.
}
%------------------------------------------------

\vspace{1em} % Extra space between major sections





%----------------------------------------------------------------------------------------
%	WORK EXPERIENCE
%----------------------------------------------------------------------------------------

\noindent\spacedlowsmallcaps{Work Experience}\vspace{1em}

\NewEntry{Summer 2015}{Research Intern, \textsc{Microsoft Research}}

\Description{\MarginText{MSR-NYC} I interned with Fernando Diaz at Microsoft
    Research in New York City, continuing our collaboration on streaming 
    news summarization. I developed scalable summarization systems 
    to provide users with brief updates of news events as they were unfolding. 
    Our work was submitted to the Temporal Summarization Track of the 2015
    Text Retrieval Conference, where we were a top performer and invited
to give a talk.}


\NewEntry{Spring 2014}{Teacher's Assistant, \textsc{Columbia University}}

\Description{\MarginText{Columbia University}I was the TA for the class 
\textit{Semantic Technologies in IBM Watson}, taught by IBM researcher 
\textsc{Alfio Gliozzo}. The class covered the various inner workings of the 
Jeopardy
playing computer. My responsibilities included teaching several lectures on
foundational natural language processing tasks and problems, and an overview
of the semantic web. Along with Dr. \textsc{Gliozzo}, I helped guide and 
supervise the
various student projects, one of which led to a publication at EMNLP 2014.}

%------------------------------------------------

\NewEntry{2008--2011}{Composer's Assistant, \textsc{Stimmung}  --- New York}

\Description{\MarginText{Stimmung \href{http://www.stimmung.tv}{stimmung.tv}}Performed 
audio engineering/mixing/editing and sheet music preparation for 
staff composers in a busy commercial music and sound post-production studio.
Posted and presented work to clients. Provided general office support and
correspondence. Organized and archived audio and video assets. 
Coordinated asset delivery to
clients/post-production services. 
Worked on many CLEO and Emmy award winning commercial
campaigns including several Super Bowl spots for such clients as: Coca-Cola, 
Mercedes-Benz, Kia, Levi's, and Monster.com.
In addition to commercials, I also helped produce music for several
independent films, documentaries, and television shows including 
\textit{Reagan} (HBO), \textit{The Rising: Rebuilding Ground Zero} (Discovery Communications), and \textit{Journey to the Stars} (Hayden Planetarium, 
American Museum of Natural History).  
 }

%------------------------------------------------

\NewEntry{2006-2008}{Production Director, \textsc{KXLU} --- Los Angeles}

\Description{\MarginText{KXLU 88.9FM \href{http://www.kxlu.com}{kxlu.com}}Worked with station directors and staff to plan 
concerts and events in the Los Angeles area, as well as the annual fundraiser.
Supervised implementation of a new website. Coordinated the recording and 
broadcast of all live and pre-recorded performances and interviews
at the station. Managed and researched equipment upgrades for the KXLU 
Production Studio. }

% \\ Reference: Big \textsc{Mike}\ \ +1 (000) 111 1111\ \ $\cdotp$\ \ \href{mailto:mike@buymore.com}{mike@buymore.com}}

%------------------------------------------------

\vspace{1em} % Extra space between major sections

%----------------------------------------------------------------------------------------
%	EDUCATION
%----------------------------------------------------------------------------------------

\spacedlowsmallcaps{Education}\vspace{1em}

\NewEntry{2014-Present}{Columbia University}

\Description{\MarginText{Doctor of Philosophy}\textit{Natural Language Processing}\ \ $\cdotp$\ \ Dept. of Computer Science\ \ $\cdotp$\ \ \newline Fu Foundation School of Engineering \& Applied Science \newline
Adviser: Prof.~\textsc{Kathleen McKeown}\\
Description: 
I am a third year Ph.D. student, working with Prof. \textsc{McKeown} on event
understanding from text data. My research has focused on 
automatic news summarization, using trainable models in 
a streaming news setting. For the past two years, I have participated in the 
Temporal Summarization Track at the Text Retrieval Conference (TREC) and have
been invited to present both times. I am generally interested in 
regression, ranking, and optimization for content selection, especially when
applied to automatic summary generation. With Prof. \textsc{McKeown} and 
\textsc{Fernando Diaz} at Microsoft Research, I have applied these techniques 
to the domain of man-made and natural disaster news.}

%With Prof. \textsc{McKeown}, I am working on event understanding from 
%newswire and social media text. Specifically, we are exploring techniques
%for automatically summarizing natural disasters, acts of terrorism, and 
%other large-scale, catastrophic events. I am currently working on the learning
%of sentence representations and similarity methods that take into account 
%event causality and correlation for use in text clustering.
%With Prof. \textsc{McKeown} and Dr. \textsc{Fernando Diaz} at Microsoft Research, I 
%participated in the Text Retrieval Conference Temporal Summarization track,
%where I experimented with regression methods for predicting important 
%information to be included in automatically generated summaries.
%I am also overseeing a masters student and two undergraduate students in an
%effort to modernize Prof. \textsc{McKeown's} automatic news summarization 
%system, Newsblaster. }

%------------------------------------------------

\NewEntry{2013-2014}{Columbia University}

\Description{\MarginText{Master of Science}GPA: 3.87\ \ $\cdotp$\ \ \textit{Natural Language Processing}\ \ $\cdotp$\ \ Dept. of Computer Science\ \ $\cdotp$\ \ \newline Fu Foundation School of Engineering \& Applied Science \newline
Adviser: Prof.~\textsc{Kathleen McKeown}\\
Description: 
I continued to pursue my interests in natural language processing, 
in addition to
machine learning and statistics. In Prof. \textsc{McKeown's} lab I worked on
question-answering (QA) for the DARPA BOLT (Broad Operational Language 
Translation) project. My research focused on unsupervised methods of 
text similarity and the application of semantic web/linked open data for QA.}


\NewEntry{2012-2013}{Columbia University}

\Description{GPA: 3.95\ \ $\cdotp$\ \ \textit{Post Baccalaureate Studies}\ \ $\cdotp$\ \ School of Continuing Education \newline
Description: 
While taking introductory courses in computer science, I also worked as a 
research assistant for Prof. \textsc{Kathleen McKeown} and her student, 
\textsc{Sara Rosenthal}.
Responsibilities included annotating research corpora for
supervised learning systems, developing web crawlers to extract user discussions from online forums, and building research corpora for studies in 
automatic influence and agreement detection in natural
language.}

\NewEntry{2011}{Baruch College, CUNY}

\Description{GPA: 4.0\ \ $\cdotp$\ \ Continuing \& Professional Studies \newline
Description: I took two classes on Java and Oracle SQL development.}

\NewEntry{2004-2008}{Loyola Marymount University}

\Description{\MarginText{Bachelor of Arts}GPA: 3.34\ \ $\cdotp$\ \ \textit{Music/Recording Arts Double Major}\ \ $\cdotp$\ \ College of Communication and Fine Arts/School of Film and Television\newline
Description: In my undergraduate degree, I pursued interests in both classical
music and sound design/mixing for film. Within the music department, I 
concetrated on
music theory/composition as well as guitar performance, culminating in two
senior theses, an original composition, \textit{String Quartet for Space
Travel}, and a guitar recital, featuring works by Antonio Lauro, Roland Dyens,
Leo Brouwer, Miguel Llobet, Antonio Vivaldi, and others. Within the film
department, I scored and sound designed/mixed many student films 
(\textit{The Cannibal Ad}, Golden Hamster(Best Overall) Award 
and 1$^{st}$ Place for Narrative Short, 2005 Northwest Projections Film 
Festival; \textit{Lily}, Best Sound, Best Film, 2007 LMU School of Film and Television ``Film Outside the
Frame Festival.'').}




%------------------------------------------------

\vspace{1em} % Extra space between major sections




%----------------------------------------------------------------------------------------
%	COMPUTER SKILLS
%----------------------------------------------------------------------------------------

\spacedlowsmallcaps{Computer Skills}\vspace{1em}

%\Description{\MarginText{Application Areas}Automatic Summarization, Text Clustering, Text Representation \& Feature Learning, Machine Learning, Data Mining, Web Scraping}

\Description{\MarginText{Languages (Adept)}English, \textsc{c/c++}, \textsc{python}, \textsc{lua}, \textsc{java}, \textsc{perl}, \textsc{html}, \LaTeX}
\Description{\MarginText{Languages (Familiar)}Latin, \textsc{matlab}, \textsc{X10}, \textsc{javascript}, \textsc{SQL}, \textsc{SPARQL}, \textsc{Linux}/\textsc{Bash}/shell scripting}

%------------------------------------------------

\vspace{1em} % Extra space between major sections

%----------------------------------------------------------------------------------------
%	OTHER INFORMATION
%----------------------------------------------------------------------------------------

\spacedlowsmallcaps{Other Information}\vspace{1em}

%\Description{\MarginText{Awards}2011\ \ $\cdotp$\ \ School of Business Postgraduate Scholarship}

%\vspace{-0.5em} % Negative vertical space to counteract the vertical space between every \Description command

%\Description{2010\ \ $\cdotp$\ \ Top Achiever Award -- Commerce}

%------------------------------------------------

%\vspace{1em}

%\Description{\MarginText{Communication Skills}2010\ \ $\cdotp$\ \ Oral Presentation at the California Business Conference}

%\vspace{-0.5em} % Negative vertical space to counteract the vertical space between every \Description command

%\Description{2009\ \ $\cdotp$\ \ Poster at the Annual Business Conference in Oregon}

%------------------------------------------------

%\vspace{1em}

%\newlength{\langbox} % Create a new length for the length of languages to keep them equally spaced
%\settowidth{\langbox}{English} % Length equals the length of "English" - if you have a longer language in your list put it here

%\Description{\MarginText{Languages}\parbox{\langbox}{\textsc{English}}\ \ $\cdotp$\ \ \ Mothertongue}

%\vspace{-0.5em} % Negative vertical space to counteract the vertical space between every \Description command

%\Description{\parbox{\langbox}{\textsc{Spanish}}\ \ $\cdotp$\ \ \ Intermediate (conversationally fluent)}

%\vspace{-0.5em} % Negative vertical space to counteract the vertical space between every \Description command

%\Description{\parbox{\langbox}{\textsc{Dutch}}\ \ $\cdotp$\ \ \ Basic (simple words and phrases only)}

%\vspace{1em} % Negative vertical space to counteract the vertical space between every \Description command

%------------------------------------------------

\Description{\MarginText{Interests}Experimental Music\ \ $\cdotp$\ \ Pop Music\ \ $\cdotp$\ \ Punk Rock}

%----------------------------------------------------------------------------------------

\end{cv}

\end{document}
